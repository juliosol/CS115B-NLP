\documentclass[11pt]{article}
\usepackage{fullpage}
\usepackage{amsfonts}
\usepackage{amssymb}
\usepackage{amsmath}
\usepackage{xcolor}
\usepackage{hyperref}

\newcommand{\F}{\mathbb{F}}
\newcommand{\np}{\mathop{\rm NP}}
\newcommand{\Z}{{\mathbb Z}}
\newcommand{\vol}{\mathop{\rm Vol}}
\newcommand{\conp}{\mathop{\rm co-NP}}
\newcommand{\atisp}{\mathop{\rm ATISP}}
\renewcommand{\vec}[1]{{\mathbf #1}}
\newcommand{\cupdot}{\mathbin{\mathaccent\cdot\cup}}
\newcommand{\mmod}[1]{\ (\mathrm{mod}\ #1)}

\setlength{\parskip}{\medskipamount}
\setlength{\parindent}{0in}
%\input{dansmacs}


\begin{document}

	\section*{CS 121 Homework 1: Fall
		2021}\label{cs-121-homework-zero-fall-2021}

\textbf{Some policies:} (See the course syllabus at  

{\tt \url{http://madhu.seas.harvard.edu/courses/Fall2021/syllabus.html}} for the
full policies.)

\begin{itemize}
\item
  {\bf Collaboration:} You can collaborate with other students that are currently enrolled in
  this course in brainstorming and thinking through approaches to
  solutions but you should write the solutions on your own: you must wait
  one hour after any collaboration or use of notes from collaboration 
  before any writing in your own solutions that you will submit.
\item
  {\bf Owning your solution:} Always make sure that you ``own'' your solutions to this other problem
  sets. That is, you should always first grapple with the problems on
  your own, and even if you participate in brainstorming sessions, make
  sure that you completely understand the ideas and details underlying
  the solution. This is in your interest as it ensures you have a solid
  understanding of the course material, and will help in the midterms
  and final. Getting 80\% of the problem
  set questions right on your own will be much better to both your
  understanding than getting 100\% of the questions through
  gathering hints from others without true understanding.
\item
  {\bf Serious violations:} Sharing questions or solutions with anyone outside this course,
  including posting on outside websites, is a violation of the honor
  code policy. Collaborating with anyone except students currently
  taking this course or using material from past years from this or
  other courses is a violation of the honor code policy.
%\item {\bf Submission Format:} The submitted PDF should be typed and in the same format and pagination as ours. Please include the text of the problems and write \textbf{Solution X:} before your solution. Please mark in gradescope the pages where the solution to each question appears. Points will be deducted if you submit in a different format.
\item {\bf Submission Format:} We strongly prefer the submitted PDF be typed and in the same format and pagination as ours. Images of handwritten answers are acceptable but not preferred. In either case, please include the text of the problems and write \textbf{Solution X:} before your solution. Please mark in gradescope the pages where the solution to each question appears. Points will be deducted if the pages are marked incorrectly. 
\item {\bf Late Day Policy:} To give students some flexibility to manage your schedule, you are allowed a net total of {\bf eight} late days through the semester, but you may not take more than {\bf two} late days on any single problem set. 
\end{itemize}

\textbf{By writing my name here I affirm that I am aware of all policies
and abided by them while working on this problem set:}

\textbf{Your name:} %(Write name here)

\textbf{Collaborators:} 

(List here names of anyone you discussed problems or ideas for solutions with)

% For homeworks 1 through n
\textbf{No. of late days used on previous psets (not including Homework Zero): 0
}\\
\textbf{No. of late days used after including this pset: %Fill this in
}


\newpage
	% For homeworks 1 through n

% Covers: Math prelims, Representations and Circuits-1


	\newpage


	\subsection*{Questions}\label{questions}

{	\color{red}
Please solve the following problems. Some of these might be harder than
the others, so don't despair if they require more time to think or you
can't do them all. Just do your best. If you don't have a proof
for a certain statement, be upfront about it. You can always explain
clearly what you are able to prove and the point at which you were
stuck. Also you can always simply write
\textbf{``I don't know''} and you will get 15 percent of the credit for
this problem. If you are stuck on this problem set, you can use Piazza to
send a private message to all staff.

}


\newcommand{\Tower}{\mathrm{Tower}}
\textbf{Problem 1:} You may use a calculator/spreadsheet for both parts of this question.

\textbf{Problem 1.1 (10 points):} The 1977 Apple II personal computer had a processor
speed of 1.023 Mhz or about $10^6$ operations per second. In 2019 the world’s fastest supercomputer performed 93
“petaflops”  or $9.3 \times 10^{16}$
basic steps per second. For each of the following running times
(as functions of the input length $n$), compute for each of the two computers how
large an input it could handle in a week of computation, if it runs
an algorithm with that running time: (a) $n$ operations, (b) $n^2$ operations, (c) $10^6 n \log_2 n$ operations, (d) $2^n$ operations and (e) $\Tower(n)$ operations, where $\Tower(0) = 1$ and $\Tower(m) = 2^{\Tower(m-1)}$. Your answers can be approximate, but you should get the two most significant decimal digits right.

\textbf{Solution 1.1:} %(write your solution here)

After solving/showing work, please summarize your answers here:

\begin{center}
\begin{tabular}{ |c|c|c| }
 \hline
 Problem & Apple II length in a week & 2019 length in a week \\
 \hline
 a &  &  \\
 \hline
 b &  &  \\
 \hline
 c &  &  \\
 \hline
 d &  &  \\
 \hline
 e &  &  \\
 \hline
\end{tabular}
\end{center}

\textbf{Problem 1.2 (5 points):}
Typically the number of operations that the fastest computers can perform doubles every three years\footnote{For those comparing carefully with the previous problem: the Apple II was not the fastest computer of 1977.}.  So in 2022 we may expect computers performing 186 petaflops. How would you compare the largest input that computers can handle in 2022 vs. what they could handle in 2019 if they run an algorithm that makes the following number of operations:
(a) $n$ operations, (b) $n^2$ operations, (c) $10^6 n \log_2 n$ operations, (d) $2^n$ operations and (e) $\Tower(n)$ operations. For each case express your answer as ``The largest $n$ in 2022 (roughly) grows/shrinks by an additive/multiplicative factor of $X$ compared to 2019.'' for the best number $X$ and choice of ``additive'' and ``multiplicative'' you can determine.

\textbf{Solution 1.2:} %(write your solution here)

After solving/showing work, summarize answers here:

\begin{center}
\begin{tabular}{ |c|c|c|c| }
 \hline
 Problem & Grows or Shrinks & Multiplicative or Additive & Rough Factor \\
 \hline
 a &  &  & \\
 \hline
 b &  &  & \\
 \hline
 c &  &  & \\
 \hline
 d &  &  & \\
 \hline
 e &  &  & \\
 \hline
\end{tabular}
\end{center}



\textbf{Problem 2:}
For each pair of functions  $F$ and $G$ below, determine
which of the following relations hold: $F = O(G)$, $F = \Omega(G)$,
$F = o(G)$, and $F = \omega(G)$. (Note: The number of relations that hold may be 0 or more than 1.) {\color{red} Just state your answers. No explanations needed.}
\begin{enumerate}
	\item {\bf (6 points)} $F(n) = n$, $G(n) = 100n$.
	\item {\bf (6 points)}  $F(n) = n$ and $G(n) = \sqrt{n}$.
	\item {\bf (6 points)}  $F(n) = n \log n$, $G(n) = 2^{(\log n)^2}$.
    \item {\bf (6 points)}  $F(n) = 2^n$, $G(n) = 8^n$.
    \item {\bf (Bonus, 0 points\footnote{We won't use this question for grades. Try it if you're interested. It may be used for recommendations/TF hiring.})} $F(n) = \binom{n}{\lceil .1 n \rceil}$ and $G(n) = 2^{.5 n}$, where $\binom{n}k$ is the number of subsets of $[n]$ of size $k$, where $[n] = \{1, 2, \ldots, n\}$.
{\bf Hint:} You may use Stirling's approximation for this.
\end{enumerate}

\textbf{Solution 2:} %(write your answers here)



\textbf{Problem 3:}

\textbf{Problem 3.1 (8 points):}

A company named SuperParts sells four classes of products. Products of Class A (of which there are at most 32) are described by a 5-bit product code, Class B by a 4-bit code, and Classes C and D each by 2-bit codes. Design a class-name coding scheme $E:\{A,B,C,D\} \to \{0,1\}^*$ such that, if you prepend the class-name coding to the product code, the result is a 6-bit encoding of every product sold by SuperParts. (You should ensure that all products have distinct encodings.)

\textbf{Solution 3.1:} %(write your answers here)

\textbf{Problem 3.2 (8 points):}

Prove that if $E'$ is a class-name coding scheme satisfying the requirements of the previous problem part, then $E'$ is a prefix-free code. (You may use the fact that for every class there is a product associated with every product code within that class. So, e.g.,  there is a Class A product with code 00010, and there is a Class C product with code 10, a Class D product with code 10, etc.)

\textbf{Solution 3.2:} %(write your answers here)



\textbf{Problem 4:}

\newcommand{\N}{\mathbb{N}}

\textbf{Problem 4.1 (10 points):}
Show that there exists a string
representation of directed graphs with vertex set $[n]$ and at most
 $10n$ edges that uses at most $1000n \log n$ bits.

More formally: Define, for every $n,m \in \N$, $G_{n,\le m}$ to be the set of directed graphs\footnote{In CS 121, unless specified otherwise, every graph is \emph{simple}: each edge has two \emph{distinct} vertices and no two edges are equal.} over the vertex
set $[n]$ with {\em at most} $m$ edges. Prove that for
every sufficiently large $n$, there exists a one-to-one function $E:{\color{red}G_{n, \leq 10n}} \to \{0,1\}^{1000n\log n}$.

\textbf{Note:} When you see a ``round'' constant such as \(10\),
\(100\), or \(1000\) in a problem set, it usually means that it was
chosen arbitrarily, and there is no particular significance to the
actual number. In particular, it may well be that you could come up with
such a scheme where \(E\) maps \({\color{red}G_{n, \leq 10n}}\) to a string of
length at most \(cn\log n\) for some constant \(c\) that is significantly
smaller than \(1000\).

\textbf{Solution 4.1:} %(write your solution here)

\textbf{Problem 4.2 (10 points):}  Define $S_n$ to be the
set of one-to-one and onto functions mapping $[n]$ to $[n]$.
%Define $G_{2n,n}$ to be the set of directed graphs with vertex set $[2n]$ and exactly $n$ edges.
Prove that there is a one-to-one mapping from $S_n$ to ${\color{red}G_{2n,\leq n}}$.

\textbf{Solution 4.2:} %(write your solution here)

\textbf{Problem 4.3 (10 points):}
Show that the encoding length of Problem 4.1 can not be improved to $o(n \log n)$. Specifically, prove that
for every sufficiently large $n\in\N$, there is no one-to-one function
$E:{\color{red} G_{n, \leq 10n}} \to \{0,1\}^{.0001 n \log n}$.

\textbf{Solution 4.3:} %(write your solution here)

\newcommand{\Odd}{\mathrm{Odd}}
\newcommand{\OR}{\mathrm{OR}}


\textbf{Problem 5 (15 points):}
Let $\Odd_n:\{0,1\}^{2n} \to \{0,1\}$ be the function given by $\Odd_n(x_0,\ldots,x_{2n-1}) = 1$ if there exists an odd $i \in [2n]$ such that $x_i = 1$ and $x_j = 0$ for every $j <i$. Otherwise $\Odd_n(x_0,\ldots,x_{2n-1}) = 0$. Show that $\Odd_n$ has $O(n)$-sized NAND circuits. 

\textbf{Hint:} You might first want to prove that the function $\OR_n(x_0,\ldots,x_{\color{red} n-1})$ has $O(n)$ sized circuits, where $\OR_n$ is given by $\OR_n(x_0,\ldots,x_{\color{red} n-1})=1$ if there exists $i$ such that $x_i = 1$ and $\OR_n(x_0,\ldots,x_{\color{red} n-1})=0$ otherwise. Then give a recursive way to compute $\Odd_n$ in terms of $\Odd_{n-1}$ and some $\OR_?$ function.

\textbf{Solution 5:} %(write your answers here)





\end{document}
